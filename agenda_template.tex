\documentclass[11pt,a4paper]{article} 

\usepackage[english]{babel} %needs to specified for minutes package (else it will be in German)
\usepackage{a4wide}%For a wider spacing of the text (smaller left/right margin)
\usepackage{amssymb}% http://ctan.org/pkg/amssymb  for the check and cross symbol
\usepackage{pifont}% http://ctan.org/pkg/pifont    for the check and cross symbol
\usepackage{setspace}
\usepackage{minutes}

\pagestyle{plain}

%more memmorable commands to make the checked and crossed symbols
\newcommand{\done}{\ding{51}}%
\newcommand{\fail}{\ding{55}}%




%%%%%%%%%%%%%%%%%%%%%%%%%%%%%%%%%%%%%%%%%%%%%%%%%%%%%%%%%%%%%%%%%%%%%%%%%%%%%%%
%
% Important: This template compiles without errors
% always check errors, also the yellow ones, even if you get a PDF
%
%%%%%%%%%%%%%%%%%%%%%%%%%%%%%%%%%%%%%%%%%%%%%%%%%%%%%%%%%%%%%%%%%%%%%%%%%%%%%%%%



\begin{document}
\begin{Minutes}{Minutes of: Title meeting}


%Add relevant date, time and location here
\minutesdate{03-06-2023} %Write the date of when you finish the minutes
\starttime{09:00}
\endtime{10:00}
\location{Room number/location}

%Add relevant names here
\participant{Names of attendees} 
\minutetaker{Name minute taker}
%The minute taker makes notes during the meetings. This is demanding and makes it harder to participate.
%The minute taker should make concise and clear notes and rewrite it afterwards so it is readable and understandable
%to someone who did not attend the meeting. In some cases this does not have to be verbatim, as long as te main arguments
%doubts and conclusions are clear. The minute taker should ask questions if something is unclear
\moderation{Name chair} 
%The chair is in charge of the meeting. This does not mean that the chair explains and discusses everything (this is actually considered to be impolite).
%The expert on the topic should explain the situation and elaboreate. 
%Instead, the chair makes sure that everyone can speak, keeping an eye on who wants to speak and in which order
%people speak in case multiple people have something to say. The chair makes sure everyone feels inclueded and heard.
%The chair also keeps an eye on the time such that the most the important topics are discussed. 
%We recommend during/after long discussions the chair also gives some time to the minute taker to finalise/summarise the notes or ask questions.
\missingExcused{None} %In case people are not present


\maketitle



\newpage

\section{Announcements} %1-5 minutes
%The chair asks everyone if there are some announcements that might be of interest for the meeting
%Maybe someone needs to leave sooner due to another appointment
Supervisor needs to leave 10 minutes earlier due to a conference.

\section{Approval of agenda} %1-2minutes
%The chair checks with everyone if they checked the agenda for today and if something is missing
%In some cases everyone can agree to change the order of topics
%For example, if someone needs to leave earlier but their input is important on a certain topic
%The chair might also mention which topics are most important or will take more time
No comments

\section{Approve previous minutes} %1-5 minutes
%At this stage a brief summary of the previous meeting is given and the minutes are approved. 
%If there are any mistakes or important comments note them here.
The previous minutes are approved, no comments

\section{Previous Tasks}%1-2 minutes
%The minute taker of this meeting should prepare the list of tasks (except checks and crosses) 
%BEFORE the meeting. This way they can quickly be checked at this stage of the meeting.
%Re-assign tasks if someone did not do a task but still has to do it.

\begin{itemize}
\item{Everyone: last week, \newline read the literature provided for the project, \done}
\item{Student X: today, \newline bring cake to make a meeting nicer, \fail (forgot)}
\end{itemize}

\task{Student X}[next meeting]{Bring cake to the meeting}




\section{Topic 1}%Change Topic 1 to a more instructive title
%This is the first major topic on the agenda for today. 
%Make concise notes during the conversations, ask questions if something is unclear and rewrite for clarity at the end of the meeting. 
%If during the discussion on this topic a task is assigned to student X, then you could use the \task command. 
%A list of tasks is automatically created at the end of the file
Some notes on the meeting \dots 
\task{Student X}[before the next meeting]{The data needs to be downloaded and merged into a single csv file}. Some more notes \dots
\task{Student Y}[7-06-2023]{Create a github page and invite collaborators}. And possibly more notes \dots

%If during the meeting another important topic comes to mind the chair makes note and, if appropriate,
%postpones this to the `remaining discussion points'. This way the topics and discussions remain clear.

\section{Topic 2}
%Another possible topic. 

\section{Remaining discussion points} %0-10 minutes
%During the meeting someone might realise they still have another point they would like to discuss
%The chair asks if someone has a point they would like to discuss or reminds the group of a 
%previously raised point (see topic 1). However, if it is a larger topic, it might be better to postpone
%to the next meeting to better prepare or have more time to go into detail.

\section{Final question round}%0-10 minutes
%The chair gives each person a moment to ask some questions. It is always good to ask each attendee specfically if they have a question
%This helps with inclusiveness and gives the possibility to quieter people to speak up

\section{Next Meeting}%1-3 minutes
%Make sure a next meeting is planned, people are busy!
%Make a note, if that is okay with the person of interest, if someone cannot attend so people
%do not forget.
The next meeting will be on Friday 5th of June between 13:00-14:00 in C2.211. 
Student X, is absent do to hospital appointment

\section{List of Tasks}
%The following command automatically generates a list of tasks if the \task command was properly used
\listoftasks

\section{Closing the meeting}
%Note the closing time of the meeting.
The chair closes the meeting at 13:37


\end{Minutes}
\end{document}


